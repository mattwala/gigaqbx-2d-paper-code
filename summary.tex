\documentclass{article}

\usepackage{booktabs}
\usepackage{caption}
\usepackage{siunitx}
\usepackage{etoolbox}
\usepackage[colorlinks]{hyperref}
\usepackage{multirow}
\usepackage[margin=1in]{geometry}
\usepackage{graphicx}

\captionsetup[table]{position=top}

\newcommand{\pfmm}{p_{\textrm{FMM}}}
\newcommand{\pqbx}{p_{\textrm{QBX}}}

\title{Experimental Results}

\sisetup{
    table-format = 1.2e-1,
    table-number-alignment = center,
    table-sign-exponent = true,
    round-mode = places,
    round-precision = 2,
    detect-weight = true,
    mode = text,
}

\newcommand{\cellcenter}[1]{\multicolumn{1}{c}{#1}}
\newcommand{\converged}[1]{\bfseries #1}
\newenvironment{resultstable}
               {
                 \begin{table}
                   \renewrobustcmd{\bfseries}{\fontseries{b}\selectfont}
               }
               {
                 \end{table}
               }

\begin{document}

\maketitle
\listoftables
\listoffigures

\begin{resultstable}
  \centering
  \caption{Green error experiments results for GIGAQBX, (c.f.\ Table~3 of paper).}
  \input{out/green-error-linf-gigaqbx.tex}
  \bigskip
  \caption{Green error experiments results for QBX FMM of~[46], (c.f.\ Table~4 of paper).}
  \input{green-error-linf-qbxfmm.tex}
\end{resultstable}

\begin{resultstable}
  \centering
  \caption{BVP error experiment (c.f.\ Table~5 of paper).}
  \input{bvp-linf.tex}
\end{resultstable}

\begin{resultstable}
  \centering
  \caption{Particle distributions experiment (c.f.\ Table~6 of paper).}
  \input{particle-distributions.tex}
\end{resultstable}

\begin{figure}
  \centering
  \includegraphics{complexity-gigaqbx.pdf}
  \caption{Complexity of GIGAQBX FMM (c.f.\ Fig.~14 of paper).}
  \label{fig:complexity-gigaqbx}
  \bigskip
  \includegraphics{complexity-qbxfmm.pdf}
  \caption{Complexity of QBX FMM (c.f.\ Fig.~15 of paper).}
  \label{fig:complexity-qbxfmm}
\end{figure}

\begin{resultstable}
  \centering
  \caption{Operation count comparison when $\pqbx = 3$; $\pfmm = 10$ for GIGAQBX
    and $\pfmm = 15$ for QBX FMM.}
  \input{complexity-summary-qbx3-qbxfmm-vs-gigaqbx.tex}
  \bigskip
  \caption{Operation count comparison when $\pqbx = 7$; $\pfmm = 15$ for GIGAQBX
    and $\pfmm = 30$ for QBX FMM.}
  \input{complexity-summary-qbx7-qbxfmm-vs-gigaqbx.tex}
\end{resultstable}

\begin{resultstable}
  \centering
  \caption{Green error for Figure~\ref{fig:complexity-gigaqbx}.}
  \input{complexity-green-errors-gigaqbx.tex}
  \bigskip
  \caption{Green error for Figure~\ref{fig:complexity-qbxfmm}.}
  \input{complexity-green-errors-qbxfmm.tex}
\end{resultstable}

\begin{resultstable}
  \centering
  \caption{Overall operation counts versus List 3 far threshold for the GIGAQBX
    FMM, for $\pqbx = 3, \pfmm = 10$.}
  \input{complexity-summary-fmm10-qbx3-gigaqbx-threshold0-vs-threshold15.tex}
  \bigskip
  \caption{Overall operation counts versus List 3 far threshold for the GIGAQBX
    FMM, for $\pqbx = 7, \pfmm = 15$.}
  \input{complexity-summary-fmm15-qbx7-gigaqbx-threshold0-vs-threshold15.tex}
\end{resultstable}

\begin{resultstable}
  \centering
  \caption{Wall time for evaluating single-layer potential using QBX order~3,
    using $\pfmm=10$ for GIGAQBX and $\pfmm=15$ for QBX FMM.}
  \input{wall-time-summary-qbx3.tex}
  \bigskip
  \caption{Wall time for evaluating single-layer potential using QBX order~7,
    with GIGAQBX $\pfmm=15$ and QBX FMM parameters $\pfmm=30$.}
  \input{wall-time-summary-qbx7.tex}
\end{resultstable}

\begin{resultstable}
  \centering
  \caption{Summary of changes to operation counts for GIGAQBX
    (c.f. Figure~\ref{fig:complexity-gigaqbx}).}
  \input{complexity-summary-gigaqbx-old-vs-new.tex}
  \bigskip
  \caption{Summary of changes to operation counts for the QBX FMM
    (c.f. Figure~\ref{fig:complexity-qbxfmm}).}
  \input{complexity-summary-qbxfmm-old-vs-new.tex}
\end{resultstable}

\end{document}
